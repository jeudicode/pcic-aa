\documentclass[letterpaper,12pt]{article}
\usepackage[utf8]{inputenc}
\usepackage[spanish,es-tabla]{babel}
\decimalpoint
\usepackage{amsfonts}

\usepackage{mathptmx}
%\usepackage{heuristica}
%\usepackage[heuristica,vvarbb,bigdelims]{newtxmath}

\usepackage[T1]{fontenc}
\renewcommand*\oldstylenums[1]{\textosf{#1}}
\usepackage[margin=1.3in]{geometry}
\usepackage{amsthm}
\usepackage{marvosym}
\usepackage{bm}

\renewcommand\qedsymbol{\Squarepipe}

\theoremstyle{definition}
\newtheorem{definition}{Definición}[section]
\newtheorem*{thm}{Teorema}


\setlength\parindent{0pt}

\newcounter{paragraphnumber}
\newcommand{\para}{%
  \vspace{10pt}\noindent{\bfseries\refstepcounter{paragraphnumber}\theparagraphnumber.\quad}%
}

%\setsecheadstyle{\large\bfseries}
%\setsubsecheadstyle{\bfseries}

\setlength\parindent{0pt}

\pagenumbering{gobble}


\usepackage{enumitem}
\setlist{nosep}

\usepackage{xcolor}

\usepackage{hyperref}
\hypersetup{
  colorlinks,
  linkcolor={red!50!black},
  citecolor={blue!50!black},
  urlcolor={green!50!black}
}

\usepackage{amssymb}
\usepackage{amsmath}

\begin{document}



\begin{center}
  {\large Aprendizaje Automatizado}\\
  \vspace{0.2cm}
  {\large\bfseries Tarea 2}\\
  \vspace{0.2cm}
  {\large PCIC - UNAM}\\
  \vspace{0.5cm}
  {\itshape 18 de marzo de 2020}\\
  \vspace{0.5cm}
  Diego de Jesús Isla López\\
  (\href{mailto:dislalopez@gmail.com}{\itshape dislalopez@gmail.com})\\
  (\href{mailto:diego.isla@comunidad.unam.mx}{\itshape diego.isla@comunidad.unam.mx})\\
\end{center}


\section*{Ejercicio 1}

Para ambos estimadores se tomarán distribuciones normales para los atributos \textbf{estatura} y \textbf{peso}, así como una distribución categórica para el atributo \textbf{nombre}. Para las clase \textbf{género} (M, F), se toma una distribución Bernoulli.\\


\subsection*{Estimador de máxima verosimilitud (EMV)}

Dado que para los atributos \textbf{estatura} y \textbf{peso} se toma una distribución normal, tenemos:

\begin{equation}
  \hat{\mu}_{EMV} = \frac{1}{n} \cdot \sum_{i=1}^n x^{(i)}
\end{equation}

\begin{equation}
  \sigma^2_{EMV} = \frac{1}{n} \cdot \sum_{i=1}^n (x^{(i)} - \hat{\mu}_{EMV})^2
\end{equation}

Para el atributo \textbf{nombre} se calcula 

\begin{equation}
  \hat{q}_k = \frac{1}{n} \cdot c_k
\end{equation}

Para la clase \textbf{M} tenemos:\\
\begin{itemize}
  \item Estatura:
  \begin{equation}
    \hat{\mu}_{M_{estatura}} = \frac{1}{7} \cdot \sum_{i=1}^7 x^{(i)} = \frac{1}{7} \cdot (12.37) = 1.7671
  \end{equation}
  \begin{equation}
    \sigma^2_{M_{estatura}} = \frac{1}{7} \cdot \sum_{i=1}^7 (x^{(i)} - \hat{\mu}_{M_{estatura}})^2 = \frac{1}{7} \cdot (0.0137) = 0.0019
  \end{equation}
  \item Peso:
  \begin{equation}
    \hat{\mu}_{M_{peso}} = \frac{1}{7} \cdot \sum_{i=1}^7 x^{(i)} = \frac{1}{7} \cdot (547.4) = 78.2
  \end{equation}
  \begin{equation}
    \sigma^2_{M_{peso}} = \frac{1}{7} \cdot \sum_{i=1}^7 (x^{(i)} - \hat{\mu}_{M_{peso}})^2 = \frac{1}{7} \cdot (110.3599) = 15.7657
  \end{equation}
  \item Nombre:
  \begin{equation}
    \hat{q}_{Denis} = \frac{1}{7}
  \end{equation}
  \begin{equation}
    \hat{q}_{Alex} = \frac{2}{7}
  \end{equation}
  \begin{equation}
    \hat{q}_{Cris} = \frac{1}{7}
  \end{equation}
  \begin{equation}
    \hat{q}_{Juan} = \frac{2}{7}
  \end{equation}
  \begin{equation}
    \hat{q}_{Guadalupe} = \frac{1}{7}
  \end{equation}
  \begin{equation}
    \hat{q}_{Rene} = \frac{0}{7}
  \end{equation}
\end{itemize}

Para la clase \textbf{F} tenemos:\\
\begin{itemize}
  \item Estatura:
  \begin{equation}
    \hat{\mu}_{F_{estatura}} = \frac{1}{6} \cdot \sum_{i=1}^6 x^{(i)} = \frac{1}{6} \cdot (9.7098) = 1.6183
  \end{equation}
  \begin{equation}
    \sigma^2_{F_{estatura}} = \frac{1}{6} \cdot \sum_{i=1}^6 (x^{(i)} - \hat{\mu}_{F_{estatura}})^2 = \frac{1}{6} \cdot (0.1344) = 0.0224
  \end{equation}
  \item Peso:
  \begin{equation}
    \hat{\mu}_{F_{peso}} = \frac{1}{6} \cdot \sum_{i=1}^6 x^{(i)} = \frac{1}{6} \cdot (351.9) = 58.65
  \end{equation}
  \begin{equation}
    \sigma^2_{F_{peso}} = \frac{1}{6} \cdot \sum_{i=1}^6 (x^{(i)} - \hat{\mu}_{F_{peso}})^2 = \frac{1}{6} \cdot (426.0546) = 71.0091
  \end{equation}
  \item Nombre:
  \begin{equation}
    \hat{q}_{Denis} = \frac{1}{6}
  \end{equation}
  \begin{equation}
    \hat{q}_{Alex} = \frac{1}{6}
  \end{equation}
  \begin{equation}
    \hat{q}_{Cris} = \frac{1}{6}
  \end{equation}
  \begin{equation}
    \hat{q}_{Juan} = \frac{0}{6}
  \end{equation}
  \begin{equation}
    \hat{q}_{Guadalupe} = \frac{2}{6}
  \end{equation}
  \begin{equation}
    \hat{q}_{Rene} = \frac{1}{6}
  \end{equation}
\end{itemize}

Los parámetros de la clase \textbf{género} los obtenemos mediante:

\begin{equation}
  \hat{q}_{k} = \frac{N_k}{N}
\end{equation}

Entonces, para \textbf{M} tenemos:\\
\begin{equation}
  \hat{q}_{M} = \frac{7}{13}
\end{equation}

Para \textbf{F} tenemos:
\begin{equation}
  \hat{q}_{F} = \frac{6}{13}
\end{equation}

Dado que los atributos son independientes, la probabilidad en cada clase se calculará como:

\begin{equation}
  P(F|x) = P(F) \cdot P(x_{nombre}|F) \cdot P(x_{estatura}|F) \cdot P(x_{peso}|F)
\end{equation}
\begin{equation}
  P(M|x) = P(M) \cdot P(x_{nombre}|M) \cdot P(x_{estatura}|M) \cdot P(x_{peso}|M)
\end{equation}

La probabilidad de los atributos con distribución normal se calcula como:
\begin{equation}
  L(\mu,\sigma^2|x) = \frac{1}{\sqrt{2 \pi \sigma^2}} \cdot e^{\frac{-(x^{(i)}-\mu)^2}{2 \sigma^2}}
\end{equation}

Para los atributos con distribución categórica y Bernoulli, se utilizan sus respectivos valores de \(q_k\).\\

Utilizado el estimador para el primer caso \(x_1 = \) (Rene, 1.68, 65), tenemos:
\begin{itemize}
  \item Probabilidad para \textbf{F}:
  \begin{equation}
    P(F|x_1) = \frac{6}{13}  \cdot \frac{1}{6} \cdot (3.4113) \cdot (0.0341) = 0.00894 = 0.89\%
  \end{equation}
  \item Probabilidad para \textbf{M}: Dado que la probabilidad para \(\hat{q}_{Rene} = 0\) para la clase \textbf{M}, la probabilidad es 0.
\end{itemize}
\medskip

Entonces, el resultado de estimador para \(x_1\) es \textbf{F}.\\

Para el caso \(x_2 = \) (Guadalupe, 1.75, 80), tenemos:
\begin{itemize}
  \item Probabilidad para \textbf{F}:
  \begin{equation}
    P(F|x_2) = \frac{6}{13}  \cdot \frac{1}{6} \cdot (1.5998) \cdot (0.0229) = 0.0007 = 0.07\%
  \end{equation}
  \item Probabilidad para \textbf{M}: 
  \begin{equation}
    P(M|x_2) = \frac{7}{13}  \cdot \frac{1}{7} \cdot (7.7204) \cdot (0.0851) = 0.0505 = 5.05\%
  \end{equation}
\end{itemize}
\medskip

Entonces, el resultado de estimador para \(x_2\) es \textbf{M}.\\

Para el caso \(x_3 = \) (Denis, 1.80, 79), tenemos:\\
\begin{itemize}
  \item Probabilidad para \textbf{F}:
  \begin{equation}
    P(F|x_3) = \frac{6}{13}  \cdot \frac{1}{6} \cdot (0.6658) \cdot (0.0038) = 0.0001 = 0.01\%
  \end{equation}
  \item Probabilidad para \textbf{M}: 
  \begin{equation}
    P(M|x_3) = \frac{7}{13}  \cdot \frac{1}{7} \cdot (6.5352) \cdot (0.0914) = 0.0459 = 4.59\%
  \end{equation}
\end{itemize}
\medskip

Entonces, el resultado de estimador para \(x_3\) es \textbf{M}.\\

Para el caso \(x_4 = \) (Alex, 1.90, 85), tenemos:\\
\begin{itemize}
  \item Probabilidad para \textbf{F}:
  \begin{equation}
    P(F|x_4) = \frac{6}{13}  \cdot \frac{1}{6} \cdot (0.0498) \cdot (0.0007) = 0.000002 = 0.0002\%
  \end{equation}
  \item Probabilidad para \textbf{M}: 
  \begin{equation}
    P(M|x_4) = \frac{7}{13}  \cdot \frac{1}{7} \cdot (0.1943) \cdot (0.0264) = 0.0007 = 0.07\%
  \end{equation}
\end{itemize}
\medskip

Entonces, el resultado de estimador para \(x_4\) es \textbf{M}.\\

Para el caso \(x_5 = \) (Cris, 1.65, 70), tenemos:\\
\begin{itemize}
  \item Probabilidad para \textbf{F}:
  \begin{equation}
    P(F|x_5) = \frac{6}{13}  \cdot \frac{1}{6} \cdot (3.9898) \cdot (0.0202) = 0.0062 = 0.62\%
  \end{equation}
  \item Probabilidad para \textbf{M}: 
  \begin{equation}
    P(M|x_5) = \frac{7}{13}  \cdot \frac{1}{7} \cdot (0.4472) \cdot (0.0149) = 0.0005 = 0.05\%
  \end{equation}
\end{itemize}
\medskip

Entonces, el resultado de estimador para \(x_5\) es \textbf{F}.\\

\subsection*{Estimador máximo a posteriori (MAP)}

Para este estimador se tomará un valor de \(\alpha = 2\) para el atributo \textbf{nombre} en ambas clases.\\

En el caso de los atributos con distribución normal (\textbf{estatura}, \textbf{peso}), el estimador se calcula como:

\begin{equation}
  \hat{\mu} = \frac{\sigma_0^2 (\sum_{i=1}^n x^{(i)} + \sigma^2 \mu_0)}{\sigma^2_0 \cdot n + \sigma^2}
\end{equation}

donde \(\sigma^2_0\) y \(\sigma^2\) se conocen de antemano.\\

Para el atributo categórico \textbf{nombre}, el estimador se calcula como:

\begin{equation}
  \hat{q}_k = \frac{c_k + a_k - 1}{n + \sum_{k=1}^K - K} 
\end{equation}

donde \(K\) es el número total de clases del atributo (6) y \(n\) es el número de elementos de la clase (para \textbf{M} o \textbf{F}).\\

Finalmente, el estimador para la clase \textbf{género} se obtiene mediante:

\begin{equation}
  \hat{q}_k = \frac{N_k + \alpha_k - 1}{N + \beta_k + \alpha_k - 2 }
\end{equation}

donde \(\beta_k\) es el número de elementos de la clase.\\


Para la clase \textbf{M} tenemos:\\

\begin{itemize}
  \item Estatura:
  \begin{itemize}
    \item \(\mu_0 = 1.7\)
    \item \(\sigma^2_0 = 0.3\)
    \item \(\sigma^2 = 0.0020\)
  \end{itemize}
  
  Entonces:\\
  
  \begin{equation}
    \hat{\mu}_{M_{estatura}} = \frac{(0.3)(12.37) + (0.0020)(1.7)}{(7)(0.3) + 0.0020} = 1.767
  \end{equation}
  \item Peso:
  \begin{itemize}
    \item \(\mu_0 = 85.5\)
    \item \(\sigma^2_0 = 17.0\)
    \item \(\sigma^2 = 15.76\)
  \end{itemize}
  
  Entonces:\\
  
  \begin{equation}
    \hat{\mu}_{M_{peso}} = \frac{(17)(547.4) + (15.76)(85.5)}{(17)(7) + 15.76} = 79.0537
  \end{equation}
  \item Nombre:
  \begin{equation}
    \hat{q}_{Denis} = \frac{1+2-1}{7-6+12} = \frac{2}{13} 
  \end{equation}
  \begin{equation}
    \hat{q}_{Guadalupe} = \frac{1+2-1}{7-6+12} = \frac{2}{13} 
  \end{equation}
  \begin{equation}
    \hat{q}_{Alex} = \frac{2+2-1}{7-6+12} = \frac{3}{13} 
  \end{equation}
  \begin{equation}
    \hat{q}_{Cris} = \frac{1+2-1}{7-6+12} = \frac{2}{13} 
  \end{equation}
  \begin{equation}
    \hat{q}_{Juan} = \frac{2+2-1}{7-6+12} = \frac{3}{13} 
  \end{equation}
  \begin{equation}
    \hat{q}_{Rene} = \frac{0+2-1}{7-6+12} = \frac{1}{13} 
  \end{equation}
\end{itemize}

Para la clase \textbf{F} tenemos:\\

\begin{itemize}
  \item Estatura:
  \begin{itemize}
    \item \(\mu_0 = 1.5\)
    \item \(\sigma^2_0 = 0.1\)
    \item \(\sigma^2 = 0.0074\)
  \end{itemize}
  
  Entonces:
  
  \begin{equation}
    \hat{\mu}_{F_{estatura}} = \frac{(0.1)(9.7098) + (0.0074)(1.5)}{(6)(0.1) + 0.0074} = 1.6168
  \end{equation}
  \item Peso:
  \begin{itemize}
    \item \(\mu_0 = 70.3\)
    \item \(\sigma^2_0 = 85.0\)
    \item \(\sigma^2 = 71.0\)
  \end{itemize}
  
  Entonces:
  
  \begin{equation}
    \hat{\mu}_{F_{peso}} = \frac{(85)(351.9) + (71)(70.3)}{(6)(85) + 71} = 60.0736
  \end{equation}
  \item Nombre:
  \begin{equation}
    \hat{q}_{Denis} = \frac{1+2-1}{6-6+12} = \frac{1}{6} 
  \end{equation}
  \begin{equation}
    \hat{q}_{Guadalupe} = \frac{2+2+1}{6-6+12} = \frac{1}{4} 
  \end{equation}
  \begin{equation}
    \hat{q}_{Alex} = \frac{1+2-1}{6-6+12} = \frac{1}{6} 
  \end{equation}
  \begin{equation}
    \hat{q}_{Cris} = \frac{1+2-1}{6-6+12} = \frac{1}{6} 
  \end{equation}
  \begin{equation}
    \hat{q}_{Juan} = \frac{0+2-1}{6-6+12} = \frac{1}{12} 
  \end{equation}
  \begin{equation}
    \hat{q}_{Rene} = \frac{1+2-1}{6-6+12} = \frac{1}{6} 
  \end{equation}
\end{itemize}

Para la clase \textbf{género}, tenemos:\\

\begin{equation}
  \hat{q}_F = \frac{6+2-1}{13+2+2-2} = \frac{7}{15}
\end{equation}

\begin{equation}
  \hat{q}_F = \frac{7+2-1}{13+2+2-2} = \frac{8}{15}
\end{equation}

\medskip


Utilizado el estimador para el primer caso \(x_1 = \) (Rene, 1.68, 65), tenemos:\\
\begin{itemize}
  \item Probabilidad para \textbf{F}:
  \begin{equation}
    P(F|x_1) = \frac{7}{15}  \cdot \frac{1}{6} \cdot (3.5434) \cdot (0.0399) = 0.0109 = 1.09\%
  \end{equation}
  \item Probabilidad para \textbf{M}:
  \begin{equation}
    P(M|x_1) = \frac{8}{15}  \cdot \frac{1}{13} \cdot (1.34) \cdot (0.0001) = 0.00001 = 0.001\%
  \end{equation}
\end{itemize}
\medskip

Entonces, el resultado de estimador para \(x_1\) es \textbf{F}.\\

Para el caso \(x_2 = \) (Guadalupe, 1.75, 80), tenemos:\\
\begin{itemize}
  \item Probabilidad para \textbf{F}:
  \begin{equation}
    P(F|x_2) = \frac{7}{15}  \cdot \frac{1}{4} \cdot (1.4008) \cdot (0.0028) = 0.0004 = 0.04\%
  \end{equation}
  \item Probabilidad para \textbf{M}: 
  \begin{equation}
    P(M|x_2) = \frac{8}{15}  \cdot \frac{2}{13} \cdot (8.2932) \cdot (0.0976) = 0.0646 = 6.46\%
  \end{equation}
\end{itemize}
\medskip

Entonces, el resultado de estimador para \(x_2\) es \textbf{M}.\\

Para el caso \(x_3 = \) (Denis, 1.80, 79), tenemos:\\
\begin{itemize}
  \item Probabilidad para \textbf{F}:
  \begin{equation}
    P(F|x_3) = \frac{7}{15}  \cdot \frac{1}{6} \cdot (0.4813) \cdot (0.0037) = 0.0001 = 0.01\%
  \end{equation}
  \item Probabilidad para \textbf{M}: 
  \begin{equation}
    P(M|x_3) = \frac{8}{15}  \cdot \frac{2}{13} \cdot (6.8033) \cdot (0.1004) = 0.056 = 5.6\%
  \end{equation}
\end{itemize}
\medskip

Entonces, el resultado de estimador para \(x_3\) es \textbf{M}.\\

Para el caso \(x_4 = \) (Alex, 1.90, 85), tenemos:\\
\begin{itemize}
  \item Probabilidad para \textbf{F}:
  \begin{equation}
    P(F|x_4) = \frac{7}{15}  \cdot \frac{1}{6} \cdot (0.0206) \cdot (0.0005) = 0.0000009 = 0.00009\%
  \end{equation}
  \item Probabilidad para \textbf{M}: 
  \begin{equation}
    P(M|x_4) = \frac{8}{15}  \cdot \frac{3}{13} \cdot (0.1076) \cdot (0.0327) = 0.0004 = 0.04\%
  \end{equation}
\end{itemize}
\medskip

Entonces, el resultado de estimador para \(x_4\) es \textbf{M}.\\

Para el caso \(x_5 = \) (Cris, 1.65, 70), tenemos:\\
\begin{itemize}
  \item Probabilidad para \textbf{F}:
  \begin{equation}
    P(F|x_5) = \frac{7}{15}  \cdot \frac{1}{6} \cdot (4.3065) \cdot (0.0236) = 0.0079 = 0.79\%
  \end{equation}
  \item Probabilidad para \textbf{M}: 
  \begin{equation}
    P(M|x_5) = \frac{8}{15}  \cdot \frac{2}{13} \cdot (0.2898) \cdot (0.0074) = 0.0001 = 0.01\%
  \end{equation}
\end{itemize}
\medskip

Entonces, el resultado de estimador para \(x_5\) es \textbf{F}.




\section*{Ejercicio 2}

Los resultados obtenidos fueron:\\

\begin{itemize}
  \item Reportados como spam: 1500 (29\%)
  \item Reportados como no spam: 3672 (71\%)
\end{itemize}
\medskip
Se utilizaron dos clasificadores bayesianos: el primero usando una distribución multinomial para los regisros y el segundo usando una distribución Bernoulli, manejando los datos como incidencia de palabras en lugar de número de apariciones.\\

El clasificador multinomial obtuvo un 95\% de precisión en la predicción sobre el conjunto de entrenamiento y un 94\% sobre el conjunto de pruebas. A su vez, el clasificador Bernoulli obtuvo un 86\% de precisión en la predicción sobre el conjunto de entrenamiento y un 84\% sobre el conjunto de pruebas. Esto puede indicarnos utilizar ambos enfoques para un análisis de textos pudiera llegar a ser adecuado; sin embargo, es clara la ventaja que conlleva el utilizar una distribución multinomial.

\section*{Ejercicio 3}

Se utilizaron tres conjuntos de prueba. Cada uno de ellos fue manipulado para completar los datos faltantes en tres formas: utilizando la media, la mediana y la moda. En los tres casos el clasificador obtuvo resultados de precisión del 100\% tanto con el conjunto de prueba como con el conjunto de entrenamiento. Esto pudiera indicar que el total de datos pudiera ser pequeño para el problema que se quiere resolver.

\end{document}

